\documentclass{article}
\usepackage{graphicx} % Required for inserting images

\begin{titlepage}
\begin{center}
\vspace{3cm}
{\scshape\Huge \textbf{Algoritmo actividad 5 PUC} \par}
\rule{80mm}{0.1mm} \\
\textbf{Juan Francisco Navarrete}

\textbf{Mayo 2025}
\end{center}
\begin{document}

\vspace{4cm}
\section*{Índice}
\Large
\Large
1. Algoritmo de ordenamiento bubble sort \\
1.1 Quien creo el algoritmo \\
1.2 De donde lo sacaron
1.3 Para que se utiliza


\newpage
\section{Algoritmo de ordenamiento bubble sort}
\vspace{2cm}
\subsection{Quien creo el algoritmo}
\normalsize
\textbf{ Edward Harry Friend nacio el 14 de enero de 1929 y fallecio el  5 de marzo de 2019. Fue un destacado matemático y actuario estadounidense, reconocido tanto por sus contribuciones al desarrollo de algoritmos computacionales como por su impacto en los sistemas de pensiones del sector público en Estados Unidos.}
\vspace{1cm}
\subsection{De donde lo sacaron}
\textbf{La primera descripción del algoritmo de ordenamiento de bubble sort apareció en un artículo de 1956 del matemático y actuario Edward Harry Friend, Sorting on electronic computer systems, publicado en el tercer número del tercer volumen del Journal of the Association for Computing Machinery (ACM), como "algoritmo de intercambio de ordenamiento.}

\vspace{1cm}
\subsection{Para que se utiliza}
\textbf{El ordenamiento de bubble sort(burbuja) es un algoritmo de ordenamiento simple que recorre repetidamente una lista, compara elementos adyacentes y los intercambia si están en el orden incorrecto. El recorrido se repite hasta que no se necesitan intercambios, lo que indica que la lista está ordenada.}



\end{document}
